\documentclass{article}
\author{Короткевич Л. В., М8О-208Б-19}
\usepackage[utf8]{inputenc}
\usepackage[russian]{babel}

\usepackage{sagetex}
\setlength{\sagetexindent}{10ex}

\usepackage[left=15mm,top=30mm,bottom=30mm,right=15mm]{geometry}

\begin{document}

\begin{center}
\Large{\textbf{Лабораторная работа 2.}}
\end{center}

\section{Задание}
\begin{enumerate}
\item Привести поверхность, заданную уравнением, к каноническому виду.
\item Построить исходную поверхность и поверхность в каноническом виде.
\item Собственные числа и вектора рассчитать вручную, сравнить с результатом встроенных функций.
\end{enumerate}

\begin{center}
Вариант 5.
\end{center}
$$f = -9x^2 + 7y^2 + 8yz - 3z^2 - 4x + 9y - 10$$

\section{Построение исходной поверхности}
\begin{sagesilent}
var("x y z")
\end{sagesilent}
\begin{sageblock}
f(x, y, z) = -9*x**2 + 7*y**2 + 8*y*z - 3*z**2 - 4*x + 9*y - 10
\end{sageblock}

Выведем считанную функцию на экран:

\begin{center}
$\sage{f(x=x, y=y, z=z)}$
\end{center}

Построим исходную поверхность.

\begin{center}
\sageplot{implicit_plot3d(f(x=x, y=y, z=z), (x, -30, 10), (y, -30, 10), (z, -10, 30), figsize=3)}
\end{center}

\section{Приведение поверхности к каноническому виду}
Составим матрицу А для квадратичной формы и матрицу B, состоящую из коэффициентов квадратичной формы, линейной формы и свободного члена.

\begin{sageblock}
A = matrix([
    [-9, 0, 0],
    [0, 7, 4],
    [0, 4, -3]
])
B = matrix([
    [-9, 0, 0, -4],
    [0, 7, 4, 9],
    [0, 4, -3, 0],
    [-4, 9, 0, -10]
])
\end{sageblock}

Вычислим ортогональные инварианты.

\begin{sageblock}
trace1 = A.trace()
trace2 = A[0:2, 0:2].det() + A[[0, 2], [0, 2]].det() + A[1:3, 1:3].det()
a_det = A.det()
b_det = B.det()
\end{sageblock}

Получили следующие результаты:

\begin{center}
$trace1 = \sage{trace1}$

$trace2 = \sage{trace2}$

$a_det = \sage{a_det}$

$b_det = \sage{b_det}$
\end{center}

Найдем собственные значения матрицы А.

\begin{sagesilent}
var("ev")
\end{sagesilent}

\begin{sageblock}
E = matrix([
    [1, 0, 0],
    [0, 1, 0],
    [0, 0, 1]
])

eigen_values = []
for eigen_value in solve((A - ev * E).det() == 0, ev):
    eigen_values.append(eigen_value.rhs())
\end{sageblock}

Собственные значения:
$$\sage{eigen_values[0]}$$
$$\sage{eigen_values[1]}$$
$$\sage{eigen_values[2]}$$

СЗ в численном виде: 
$$\sage{eigen_values[0].n()}$$
$$\sage{eigen_values[1].n()}$$
$$\sage{eigen_values[2].n()}$$

Теперь найдем собственные значения через встроенную функцию.

\begin{sageblock}
A.eigenvalues()
\end{sageblock}

\begin{sagesilent}
ev_auto = A.eigenvalues()
\end{sagesilent}

Получили: 
$$\sage{ev_auto[0]}$$
$$\sage{ev_auto[1]}$$
$$\sage{ev_auto[2]}$$ 

Результаты вычисления собственных значений совпали.

Теперь составим каноническое уравнение поверхности. Сначала определим коэффициенты для нового уравнения.

\begin{sagesilent}
var("x1 y1 z1")
\end{sagesilent}

\begin{sageblock}
f_canonical(x1, y1, z1) = z1**2 * eigen_values[1] + x1**2 * eigen_values[2] + y1**2 * eigen_values[0] + b_det / a_det
\end{sageblock}

Получили уравнение:
$$\sage{f_canonical(x1, y1, z1)}$$

Построим полученную поверхность.

\begin{center}
\sageplot{implicit_plot3d(f_canonical(x1=x1, y1=y1, z1=z1), (x1, -30, 10), (y1, -30, 10), (z1, -10, 30), figsize=4)}
\end{center}

\end{document}